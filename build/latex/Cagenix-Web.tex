% Generated by Sphinx.
\def\sphinxdocclass{report}
\documentclass[letterpaper,10pt,english]{sphinxmanual}
\usepackage[utf8]{inputenc}
\DeclareUnicodeCharacter{00A0}{\nobreakspace}
\usepackage{cmap}
\usepackage[T1]{fontenc}
\usepackage{babel}
\usepackage{times}
\usepackage[Bjarne]{fncychap}
\usepackage{longtable}
\usepackage{sphinx}
\usepackage{multirow}


\title{Cagenix-Web Documentation}
\date{January 27, 2014}
\release{0.1a1}
\author{Jason Myers for Prime Notion Technologies}
\newcommand{\sphinxlogo}{}
\renewcommand{\releasename}{Release}
\makeindex

\makeatletter
\def\PYG@reset{\let\PYG@it=\relax \let\PYG@bf=\relax%
    \let\PYG@ul=\relax \let\PYG@tc=\relax%
    \let\PYG@bc=\relax \let\PYG@ff=\relax}
\def\PYG@tok#1{\csname PYG@tok@#1\endcsname}
\def\PYG@toks#1+{\ifx\relax#1\empty\else%
    \PYG@tok{#1}\expandafter\PYG@toks\fi}
\def\PYG@do#1{\PYG@bc{\PYG@tc{\PYG@ul{%
    \PYG@it{\PYG@bf{\PYG@ff{#1}}}}}}}
\def\PYG#1#2{\PYG@reset\PYG@toks#1+\relax+\PYG@do{#2}}

\expandafter\def\csname PYG@tok@gd\endcsname{\def\PYG@tc##1{\textcolor[rgb]{0.63,0.00,0.00}{##1}}}
\expandafter\def\csname PYG@tok@gu\endcsname{\let\PYG@bf=\textbf\def\PYG@tc##1{\textcolor[rgb]{0.50,0.00,0.50}{##1}}}
\expandafter\def\csname PYG@tok@gt\endcsname{\def\PYG@tc##1{\textcolor[rgb]{0.00,0.27,0.87}{##1}}}
\expandafter\def\csname PYG@tok@gs\endcsname{\let\PYG@bf=\textbf}
\expandafter\def\csname PYG@tok@gr\endcsname{\def\PYG@tc##1{\textcolor[rgb]{1.00,0.00,0.00}{##1}}}
\expandafter\def\csname PYG@tok@cm\endcsname{\let\PYG@it=\textit\def\PYG@tc##1{\textcolor[rgb]{0.25,0.50,0.56}{##1}}}
\expandafter\def\csname PYG@tok@vg\endcsname{\def\PYG@tc##1{\textcolor[rgb]{0.73,0.38,0.84}{##1}}}
\expandafter\def\csname PYG@tok@m\endcsname{\def\PYG@tc##1{\textcolor[rgb]{0.13,0.50,0.31}{##1}}}
\expandafter\def\csname PYG@tok@mh\endcsname{\def\PYG@tc##1{\textcolor[rgb]{0.13,0.50,0.31}{##1}}}
\expandafter\def\csname PYG@tok@cs\endcsname{\def\PYG@tc##1{\textcolor[rgb]{0.25,0.50,0.56}{##1}}\def\PYG@bc##1{\setlength{\fboxsep}{0pt}\colorbox[rgb]{1.00,0.94,0.94}{\strut ##1}}}
\expandafter\def\csname PYG@tok@ge\endcsname{\let\PYG@it=\textit}
\expandafter\def\csname PYG@tok@vc\endcsname{\def\PYG@tc##1{\textcolor[rgb]{0.73,0.38,0.84}{##1}}}
\expandafter\def\csname PYG@tok@il\endcsname{\def\PYG@tc##1{\textcolor[rgb]{0.13,0.50,0.31}{##1}}}
\expandafter\def\csname PYG@tok@go\endcsname{\def\PYG@tc##1{\textcolor[rgb]{0.20,0.20,0.20}{##1}}}
\expandafter\def\csname PYG@tok@cp\endcsname{\def\PYG@tc##1{\textcolor[rgb]{0.00,0.44,0.13}{##1}}}
\expandafter\def\csname PYG@tok@gi\endcsname{\def\PYG@tc##1{\textcolor[rgb]{0.00,0.63,0.00}{##1}}}
\expandafter\def\csname PYG@tok@gh\endcsname{\let\PYG@bf=\textbf\def\PYG@tc##1{\textcolor[rgb]{0.00,0.00,0.50}{##1}}}
\expandafter\def\csname PYG@tok@ni\endcsname{\let\PYG@bf=\textbf\def\PYG@tc##1{\textcolor[rgb]{0.84,0.33,0.22}{##1}}}
\expandafter\def\csname PYG@tok@nl\endcsname{\let\PYG@bf=\textbf\def\PYG@tc##1{\textcolor[rgb]{0.00,0.13,0.44}{##1}}}
\expandafter\def\csname PYG@tok@nn\endcsname{\let\PYG@bf=\textbf\def\PYG@tc##1{\textcolor[rgb]{0.05,0.52,0.71}{##1}}}
\expandafter\def\csname PYG@tok@no\endcsname{\def\PYG@tc##1{\textcolor[rgb]{0.38,0.68,0.84}{##1}}}
\expandafter\def\csname PYG@tok@na\endcsname{\def\PYG@tc##1{\textcolor[rgb]{0.25,0.44,0.63}{##1}}}
\expandafter\def\csname PYG@tok@nb\endcsname{\def\PYG@tc##1{\textcolor[rgb]{0.00,0.44,0.13}{##1}}}
\expandafter\def\csname PYG@tok@nc\endcsname{\let\PYG@bf=\textbf\def\PYG@tc##1{\textcolor[rgb]{0.05,0.52,0.71}{##1}}}
\expandafter\def\csname PYG@tok@nd\endcsname{\let\PYG@bf=\textbf\def\PYG@tc##1{\textcolor[rgb]{0.33,0.33,0.33}{##1}}}
\expandafter\def\csname PYG@tok@ne\endcsname{\def\PYG@tc##1{\textcolor[rgb]{0.00,0.44,0.13}{##1}}}
\expandafter\def\csname PYG@tok@nf\endcsname{\def\PYG@tc##1{\textcolor[rgb]{0.02,0.16,0.49}{##1}}}
\expandafter\def\csname PYG@tok@si\endcsname{\let\PYG@it=\textit\def\PYG@tc##1{\textcolor[rgb]{0.44,0.63,0.82}{##1}}}
\expandafter\def\csname PYG@tok@s2\endcsname{\def\PYG@tc##1{\textcolor[rgb]{0.25,0.44,0.63}{##1}}}
\expandafter\def\csname PYG@tok@vi\endcsname{\def\PYG@tc##1{\textcolor[rgb]{0.73,0.38,0.84}{##1}}}
\expandafter\def\csname PYG@tok@nt\endcsname{\let\PYG@bf=\textbf\def\PYG@tc##1{\textcolor[rgb]{0.02,0.16,0.45}{##1}}}
\expandafter\def\csname PYG@tok@nv\endcsname{\def\PYG@tc##1{\textcolor[rgb]{0.73,0.38,0.84}{##1}}}
\expandafter\def\csname PYG@tok@s1\endcsname{\def\PYG@tc##1{\textcolor[rgb]{0.25,0.44,0.63}{##1}}}
\expandafter\def\csname PYG@tok@gp\endcsname{\let\PYG@bf=\textbf\def\PYG@tc##1{\textcolor[rgb]{0.78,0.36,0.04}{##1}}}
\expandafter\def\csname PYG@tok@sh\endcsname{\def\PYG@tc##1{\textcolor[rgb]{0.25,0.44,0.63}{##1}}}
\expandafter\def\csname PYG@tok@ow\endcsname{\let\PYG@bf=\textbf\def\PYG@tc##1{\textcolor[rgb]{0.00,0.44,0.13}{##1}}}
\expandafter\def\csname PYG@tok@sx\endcsname{\def\PYG@tc##1{\textcolor[rgb]{0.78,0.36,0.04}{##1}}}
\expandafter\def\csname PYG@tok@bp\endcsname{\def\PYG@tc##1{\textcolor[rgb]{0.00,0.44,0.13}{##1}}}
\expandafter\def\csname PYG@tok@c1\endcsname{\let\PYG@it=\textit\def\PYG@tc##1{\textcolor[rgb]{0.25,0.50,0.56}{##1}}}
\expandafter\def\csname PYG@tok@kc\endcsname{\let\PYG@bf=\textbf\def\PYG@tc##1{\textcolor[rgb]{0.00,0.44,0.13}{##1}}}
\expandafter\def\csname PYG@tok@c\endcsname{\let\PYG@it=\textit\def\PYG@tc##1{\textcolor[rgb]{0.25,0.50,0.56}{##1}}}
\expandafter\def\csname PYG@tok@mf\endcsname{\def\PYG@tc##1{\textcolor[rgb]{0.13,0.50,0.31}{##1}}}
\expandafter\def\csname PYG@tok@err\endcsname{\def\PYG@bc##1{\setlength{\fboxsep}{0pt}\fcolorbox[rgb]{1.00,0.00,0.00}{1,1,1}{\strut ##1}}}
\expandafter\def\csname PYG@tok@kd\endcsname{\let\PYG@bf=\textbf\def\PYG@tc##1{\textcolor[rgb]{0.00,0.44,0.13}{##1}}}
\expandafter\def\csname PYG@tok@ss\endcsname{\def\PYG@tc##1{\textcolor[rgb]{0.32,0.47,0.09}{##1}}}
\expandafter\def\csname PYG@tok@sr\endcsname{\def\PYG@tc##1{\textcolor[rgb]{0.14,0.33,0.53}{##1}}}
\expandafter\def\csname PYG@tok@mo\endcsname{\def\PYG@tc##1{\textcolor[rgb]{0.13,0.50,0.31}{##1}}}
\expandafter\def\csname PYG@tok@mi\endcsname{\def\PYG@tc##1{\textcolor[rgb]{0.13,0.50,0.31}{##1}}}
\expandafter\def\csname PYG@tok@kn\endcsname{\let\PYG@bf=\textbf\def\PYG@tc##1{\textcolor[rgb]{0.00,0.44,0.13}{##1}}}
\expandafter\def\csname PYG@tok@o\endcsname{\def\PYG@tc##1{\textcolor[rgb]{0.40,0.40,0.40}{##1}}}
\expandafter\def\csname PYG@tok@kr\endcsname{\let\PYG@bf=\textbf\def\PYG@tc##1{\textcolor[rgb]{0.00,0.44,0.13}{##1}}}
\expandafter\def\csname PYG@tok@s\endcsname{\def\PYG@tc##1{\textcolor[rgb]{0.25,0.44,0.63}{##1}}}
\expandafter\def\csname PYG@tok@kp\endcsname{\def\PYG@tc##1{\textcolor[rgb]{0.00,0.44,0.13}{##1}}}
\expandafter\def\csname PYG@tok@w\endcsname{\def\PYG@tc##1{\textcolor[rgb]{0.73,0.73,0.73}{##1}}}
\expandafter\def\csname PYG@tok@kt\endcsname{\def\PYG@tc##1{\textcolor[rgb]{0.56,0.13,0.00}{##1}}}
\expandafter\def\csname PYG@tok@sc\endcsname{\def\PYG@tc##1{\textcolor[rgb]{0.25,0.44,0.63}{##1}}}
\expandafter\def\csname PYG@tok@sb\endcsname{\def\PYG@tc##1{\textcolor[rgb]{0.25,0.44,0.63}{##1}}}
\expandafter\def\csname PYG@tok@k\endcsname{\let\PYG@bf=\textbf\def\PYG@tc##1{\textcolor[rgb]{0.00,0.44,0.13}{##1}}}
\expandafter\def\csname PYG@tok@se\endcsname{\let\PYG@bf=\textbf\def\PYG@tc##1{\textcolor[rgb]{0.25,0.44,0.63}{##1}}}
\expandafter\def\csname PYG@tok@sd\endcsname{\let\PYG@it=\textit\def\PYG@tc##1{\textcolor[rgb]{0.25,0.44,0.63}{##1}}}

\def\PYGZbs{\char`\\}
\def\PYGZus{\char`\_}
\def\PYGZob{\char`\{}
\def\PYGZcb{\char`\}}
\def\PYGZca{\char`\^}
\def\PYGZam{\char`\&}
\def\PYGZlt{\char`\<}
\def\PYGZgt{\char`\>}
\def\PYGZsh{\char`\#}
\def\PYGZpc{\char`\%}
\def\PYGZdl{\char`\$}
\def\PYGZhy{\char`\-}
\def\PYGZsq{\char`\'}
\def\PYGZdq{\char`\"}
\def\PYGZti{\char`\~}
% for compatibility with earlier versions
\def\PYGZat{@}
\def\PYGZlb{[}
\def\PYGZrb{]}
\makeatother

\begin{document}

\maketitle
\tableofcontents
\phantomsection\label{index::doc}


Contents:


\chapter{Bid Overview}
\label{dev-bid:welcome-to-cagenix-web-s-documentation}\label{dev-bid::doc}\label{dev-bid:bid-overview}
We’re needing a web application that acts as a server-side component to a data collection mobile application. It also has a front-end for administrative purposes. This is intentionally vague thanks to NDAs, etc., but it is this “simple”. The requirements are completely solid and we don’t tolerate scope creep...


\section{Backend}
\label{dev-bid:backend}
The back-end of web app, through a JSON API, needs to:
\begin{itemize}
\item {} 
Store “activation codes” for the mobile apps associated with a person’s business. When a user downloads an app, they have to enter this code.

\item {} 
Allow a user to register using an activation code, some contact information, and other basic information. The activation code is expired but still associated with the user. The user must open an email sent to them to activate their registration.

\item {} 
Validate user’s login credentials, returning a auth token that must be passed with all JSON requests.

\item {} 
Associate with a user another type of user in a parent-child relationship. This other type of user (let’s call this a customer) just has basic information.

\item {} 
Associate a customer evaluation with a customer. This customer evaluation contains a few question–answer fields (about 20).

\item {} 
Associate a product choice with a customer. There are three product choices: “good”, “better”, “best” if you like.

\item {} 
Allow the app to retrieve a list of advertisement banners placed in specified areas of the app (think left, right, or bottom). A “hit count” for the ad banner is retained.

\end{itemize}


\section{Frontend}
\label{dev-bid:frontend}
The front-end of the web app needs to:
\begin{itemize}
\item {} 
Allow an administrator to log in.

\item {} 
Allow an administrator to create an activation code and associate very simple branding attributes with it (primary, secondary colors).

\item {} 
Allow an administrator to upload advertisement banners.

\item {} 
Allow a user to perform a password reset if they have forgotten their password. This simply allows them to log back in to the mobile app; they do not log in to the web app.

\end{itemize}

If you have any questions, let me know.


\chapter{Developer Overview}
\label{dev-overview:developer-overview}\label{dev-overview::doc}
The following document covers the internals of the Cagenix-Web
application.  Below is a list of libraries used in the application,
and the general application settings.


\section{Libraries}
\label{dev-overview:libraries}\begin{itemize}
\item {} 
Flask - Web Framework \href{http://flask.pocoo.org/}{Documentation}

\item {} 
Flask-SQLAlchemy - Simplifies the usage of SQLAlchemy within the Flask framework.
Check out the \href{http://packages.python.org/Flask-SQLAlchemy/quickstart.html\#road-to-enlightenment}{Road to Enlightenment}
for more details.  \href{http://docs.sqlalchemy.org/en/rel\_0\_8/}{SQLAlchemy Docs}

\item {} 
WTForms - Provides a flexible way of handling forms Documentation

\item {} 
Flask-WTF - Provides a simple integration with WTForms Documentation

\item {} 
Jinja2 - The prefered templating language for Flask \href{http://jinja.pocoo.org/docs/}{Documentation}

\item {} 
Flask-Security - A very powerful collection of User Authentication/Authorization/Accounting
\href{http://flask-security.readthedocs.org/en/latest/index.html}{Documentation}

\item {} 
Passlib - A stong password hashing utility \href{http://packages.python.org/passlib/new\_app\_quickstart.html\#pbkdf2}{Documentation}

\item {} 
Unicodecsv - Used to handle csv files that contain unicode characters \href{https://github.com/jdunck/python-unicodecsv}{Documentation}

\end{itemize}

It's built using the \href{https://github.com/mitsuhiko/flask/wiki/Large-app-how-to)}{Flask Large App How To} and \href{http://mattupstate.com/python/2013/06/26/how-i-structure-my-flask-applications.html?utm\_medium=referral\&utm\_source=pulsenews\#s2c}{Matt Wright's Guide}


\section{General Application Setup (\_\_init\_\_.py)}
\label{dev-overview:module-cagenix}\label{dev-overview:general-application-setup-init-py}\index{cagenix (module)}
This file contains the application setup for
SSLify, Flask-SQLAlchemy, and Flask-Security.
\index{create\_user() (in module cagenix)}

\begin{fulllineitems}
\phantomsection\label{dev-overview:cagenix.create_user}\pysiglinewithargsret{\code{cagenix.}\bfcode{create\_user}}{}{}
\end{fulllineitems}

\index{handle\_splash() (in module cagenix)}

\begin{fulllineitems}
\phantomsection\label{dev-overview:cagenix.handle_splash}\pysiglinewithargsret{\code{cagenix.}\bfcode{handle\_splash}}{\emph{*args}, \emph{**kwargs}}{}
\end{fulllineitems}

\index{logout() (in module cagenix)}

\begin{fulllineitems}
\phantomsection\label{dev-overview:cagenix.logout}\pysiglinewithargsret{\code{cagenix.}\bfcode{logout}}{}{}
This handles the logout for flask\_security and redirects to login

\end{fulllineitems}

\index{not\_found() (in module cagenix)}

\begin{fulllineitems}
\phantomsection\label{dev-overview:cagenix.not_found}\pysiglinewithargsret{\code{cagenix.}\bfcode{not\_found}}{\emph{error}}{}
returns a the 404.html template on any 404 error with in the app.
\begin{quote}\begin{description}
\item[{Parameters}] \leavevmode
\textbf{error} -- The error details

\item[{Returns}] \leavevmode
404.html

\item[{Return type}] \leavevmode
template

\end{description}\end{quote}

\end{fulllineitems}



\chapter{Models Overview}
\label{dev-models:models-overview}\label{dev-models::doc}
The following document covers the models used in the Cagenix-Web application.
\begin{itemize}
\item {} 
{\hyperref[dev-models:activation-model-label]{\emph{Activation Model}}}

\item {} 
{\hyperref[dev-models:advertisement-model-label]{\emph{Advertisement Model}}}

\item {} 
{\hyperref[dev-models:evaluation-model-label]{\emph{Evaluation Model}}}

\item {} 
{\hyperref[dev-models:patient-model-label]{\emph{Patient Model}}}

\item {} 
{\hyperref[dev-models:patientevaluationanswer-model-label]{\emph{PatientEvaluationAnswer Model}}}

\item {} 
{\hyperref[dev-models:practice-model-label]{\emph{Practice Model}}}

\item {} 
{\hyperref[dev-models:role-model-label]{\emph{Role Model}}}

\item {} 
{\hyperref[dev-models:user-model-label]{\emph{User Model}}}

\end{itemize}


\section{Activation Model}
\label{dev-models:activation-model}\label{dev-models:activation-model-label}
\begin{tabulary}{\linewidth}{|L|L|L|}
\hline
\textsf{\relax 
Field Name
} & \textsf{\relax 
Type (Length)
} & \textsf{\relax 
Description
}\\
\hline
id
 & 
Integer
 & 
Primary Key
\\

code
 & 
String(255)
 & 
Activiation Code
\\
\hline\end{tabulary}


\textbf{Virtuals}

\begin{tabulary}{\linewidth}{|L|L|L|}
\hline
\textsf{\relax 
Field Name
} & \textsf{\relax 
Type
} & \textsf{\relax 
Description
}\\
\hline
practitioner
 & 
One2One
 & 
Virtual to User/practitioner
\\

practice
 & 
Many2One
 & 
Virtual to Practice
\\
\hline\end{tabulary}



\section{Advertisement Model}
\label{dev-models:advertisement-model}\label{dev-models:advertisement-model-label}
\begin{tabulary}{\linewidth}{|L|L|L|}
\hline
\textsf{\relax 
Field Name
} & \textsf{\relax 
Type (Length)
} & \textsf{\relax 
Description
}\\
\hline
id
 & 
Integer
 & 
Primary Key
\\

name
 & 
String(80)
 & 
Banner Name
\\

asset\_location
 & 
String(80)
 & 
Location of asset or resource name
\\

position
 & 
String(80)
 & 
Screen Position
\\
\hline\end{tabulary}



\section{Evaluation Model}
\label{dev-models:evaluation-model-label}\label{dev-models:evaluation-model}
\begin{tabulary}{\linewidth}{|L|L|L|}
\hline
\textsf{\relax 
Field Name
} & \textsf{\relax 
Type (Length)
} & \textsf{\relax 
Description
}\\
\hline
id
 & 
Integer
 & 
Primary Key
\\

evaluation\_id
 & 
UUID
 & 
Used to group questions into a single
evaulation.
\\

question\_order
 & 
Integer
 & 
Position of question within an
Evaulation
\\

question\_text
 & 
Text
 & 
Text of the question
\\

answer\_one
 & 
String(255)
 & 
First answer choice
\\

answer\_two
 & 
String(255)
 & 
Second answer choice
\\

answer\_three
 & 
String(255)
 & 
Second answer choice
\\

answer\_four
 & 
String(255)
 & 
Second answer choice
\\

answer\_custom
 & 
Text
 & 
Free for all answer
\\
\hline\end{tabulary}



\section{Patient Model}
\label{dev-models:patient-model-label}\label{dev-models:patient-model}
\begin{tabulary}{\linewidth}{|L|L|L|}
\hline
\textsf{\relax 
Field Name
} & \textsf{\relax 
Type (Length)
} & \textsf{\relax 
Description
}\\
\hline
id
 & 
Integer
 & 
Primary Key
\\

first\_name
 & 
String(80)
 & 
First Name
\\

last\_name
 & 
String(80)
 & 
Last Name
\\

address\_one
 & 
String(255)
 & 
First address line
\\

address\_two
 & 
String(255)
 & 
Second address line
\\

city
 & 
String(255)
 & 
City
\\

state
 & 
String(255)
 & 
State
\\

zip\_code
 & 
String(10)
 & 
ZIP Code in 5-4 format
\\

email
 & 
String(255)
 & 
email address
\\

active
 & 
Boolean
 & 
Active Flag for archiving
\\

choice\_one
 & 
String(255)
 & 
The ``best'' choice
\\

choice\_two
 & 
String(255)
 & 
The ``better'' choice
\\

choice\_three
 & 
String(255)
 & 
The ``good'' choice
\\
\hline\end{tabulary}


\textbf{Virtuals}

\begin{tabulary}{\linewidth}{|L|L|L|}
\hline
\textsf{\relax 
Field Name
} & \textsf{\relax 
Type
} & \textsf{\relax 
Description
}\\
\hline
practitioner
 & 
Many2One
 & 
Virtual to User assigned as dentist
\\

practice
 & 
Many2One
 & 
Virtual to Practice
\\

evaulation\_answer
 & 
Many2One
 & 
Virtual to PatientEvaulationAnswer
\\
\hline\end{tabulary}



\section{PatientEvaluationAnswer Model}
\label{dev-models:patientevaluationanswer-model-label}\label{dev-models:patientevaluationanswer-model}
\begin{tabulary}{\linewidth}{|L|L|L|}
\hline
\textsf{\relax 
Field Name
} & \textsf{\relax 
Type (Length)
} & \textsf{\relax 
Description
}\\
\hline
id
 & 
Integer
 & 
Primary Key
\\

evaluation\_id
 & 
UUID
 & 
Used to group answers into a single
evaulation.
\\

answer
 & 
String(255)
 & 
Answer to question
\\

active
 & 
Boolean
 & 
Active Flag for archiving
\\
\hline\end{tabulary}


\textbf{Virtuals}

\begin{tabulary}{\linewidth}{|L|L|L|}
\hline
\textsf{\relax 
Field Name
} & \textsf{\relax 
Type
} & \textsf{\relax 
Description
}\\
\hline
question
 & 
Many2One
 & 
Question Asked
\\

patient
 & 
Many2One
 & 
User Answering Question
\\
\hline\end{tabulary}



\section{Practice Model}
\label{dev-models:practice-model-label}\label{dev-models:practice-model}
\begin{tabulary}{\linewidth}{|L|L|L|}
\hline
\textsf{\relax 
Field Name
} & \textsf{\relax 
Type (Length)
} & \textsf{\relax 
Description
}\\
\hline
id
 & 
Integer
 & 
Primary Key
\\

practice\_name
 & 
String(255)
 & 
Practice Name
\\

address\_one
 & 
String(255)
 & 
First address line
\\

address\_two
 & 
String(255)
 & 
Second address line
\\

city
 & 
String(255)
 & 
City
\\

state
 & 
String(255)
 & 
State
\\

zip\_code
 & 
String(10)
 & 
ZIP Code in 5-4 format
\\

url
 & 
String(255)
 & 
Web Address
\\

primary\_color
 & 
String(255)
 & 
Primary Branding Color
\\

secondary\_color
 & 
String(255)
 & 
Secondary Branding Color
\\

phone
 & 
String(12)
 & 
Phone
\\

active
 & 
Boolean
 & 
Active Flag for archiving
\\
\hline\end{tabulary}


\textbf{Virtuals}

\begin{tabulary}{\linewidth}{|L|L|L|}
\hline
\textsf{\relax 
Field Name
} & \textsf{\relax 
Type
} & \textsf{\relax 
Description
}\\
\hline
practitioner
 & 
One2Many
 & 
Virtual to User
\\
\hline\end{tabulary}



\section{Role Model}
\label{dev-models:role-model-label}\label{dev-models:role-model}
\begin{tabulary}{\linewidth}{|L|L|L|}
\hline
\textsf{\relax 
Field Name
} & \textsf{\relax 
Type (Length)
} & \textsf{\relax 
Description
}\\
\hline
id
 & 
Integer
 & 
Primary Key
\\

name
 & 
String(80)
 & 
Name
\\

description
 & 
String(255)
 & 
Short description of the Role's purpose
\\
\hline\end{tabulary}


\textbf{Virtuals}

\begin{tabulary}{\linewidth}{|L|L|L|}
\hline
\textsf{\relax 
Field Name
} & \textsf{\relax 
Type
} & \textsf{\relax 
Description
}\\
\hline
practitioner
 & 
One2Many
 & 
Virtual to User
\\
\hline\end{tabulary}



\section{User Model}
\label{dev-models:user-model-label}\label{dev-models:user-model}
\begin{tabulary}{\linewidth}{|L|L|L|}
\hline
\textsf{\relax 
Field Name
} & \textsf{\relax 
Type (Length)
} & \textsf{\relax 
Description
}\\
\hline
id
 & 
Integer
 & 
Primary Key
\\

first\_name
 & 
String(80)
 & 
First Name
\\

last\_name
 & 
String(80)
 & 
Last Name
\\

address\_one
 & 
String(255)
 & 
First address line
\\

address\_two
 & 
String(255)
 & 
Second address line
\\

city
 & 
String(255)
 & 
City
\\

state
 & 
String(255)
 & 
State
\\

zip\_code
 & 
String(10)
 & 
ZIP Code in 5-4 format
\\

email
 & 
String(255)
 & 
email address
\\

password
 & 
String(255)
 & 
encrypted password string
\\

secret
 & 
String(255)
 & 
Secret key used for API
\\

activation\_code
 & 
String(255)
 & 
Code used for initial account setup
\\

active
 & 
Boolean
 & 
Active Flag for archiving
\\

confirmed\_at
 & 
DateTime
 & 
UTC DateTime when user confirmed
their account.
\\

last\_login\_at
 & 
DateTime
 & 
UTC DateTime of last login
\\

current\_login\_at
 & 
DateTime
 & 
UTC DateTime of current login if active
\\

last\_login\_ip
 & 
String(255)
 & 
Last IP accessed from
\\

current\_login\_ip
 & 
String(255)
 & 
Current IP accessed from
\\

login\_count
 & 
Integer
 & 
Number of Logins
\\
\hline\end{tabulary}


\textbf{Virtuals}

\begin{tabulary}{\linewidth}{|L|L|L|}
\hline
\textsf{\relax 
Field Name
} & \textsf{\relax 
Type
} & \textsf{\relax 
Description
}\\
\hline
patients
 & 
One2Many
 & 
Virtual to Patient
\\

roles
 & 
One2Many
 & 
Virtual to Role
\\
\hline\end{tabulary}



\chapter{API Overview}
\label{dev-api-overview:api-overview}\label{dev-api-overview::doc}
The following document will walk you through using the Cagenix API
within your application.


\section{Base URL, HTTPS, and Versioning}
\label{dev-api-overview:base-url-https-and-versioning}

\subsection{Base URI}
\label{dev-api-overview:base-uri}
The Cagenix API root is located at:
\href{https://cagenix-web.herokuapp.com/api/v1/}{https://cagenix-web.herokuapp.com/api/v1/}


\subsection{HTTP and HTTPS}
\label{dev-api-overview:http-and-https}
The Cagenix API is served over HTTP and HTTPS during testing. The final
deployment will be HTTPS only, prefer to testing that method.


\subsection{Versioning}
\label{dev-api-overview:versioning}
Currently the Cagenix API does not require any type of versioning in the
requests as it is in the URI.


\subsection{Message Body}
\label{dev-api-overview:message-body}
When submitting a request to the Cagenix API with a message body, the
variables within that body should be UTF8 encoded and within a JSON object.


\section{Credentials}
\label{dev-api-overview:credentials}
In order to access the Cagenix API each developer/user must request a key and
a secret from the application admins. These two keys will be used in the
following formulas which will provide the properly formatted header values
required for all POST, PUT, PATCH, and DELETE requests. All GET requests require
the key, timestamp, and signature url parameters.


\section{GET Requests}
\label{dev-api-overview:get-requests}
A properly formed GET request includes the key timestamp, and signature url
parameters. The key is the key provided by the application admins.  The
signature is made of the HTTP-VERB + SECRET + Canonicalized Resource encoded to
UTF8 hashed via sha256 and base 64 encoded.

\begin{tabulary}{\linewidth}{|L|L|}
\hline
\textsf{\relax 
URL Param
} & \textsf{\relax 
Description
}\\
\hline
key
 & 
This is the key provided by the application admins.
\\

timestamp
 & 
This is the date and time, ISO format, used in your
header formulas. An example is 2012-10-29T01:30:20Z.
Notice the T and Z, with the T separating the date and
time and Z ending the string.
\\

signature
 & 
The header signature is a string that combines the body
hash, date and time, along with your secret, encoded to
UTF8 and hashed using sha256.
\\
\hline\end{tabulary}


\begin{notice}{note}{Note:}
Signature = URLEncode(Base64( SHA-256( UTF-8-Encoding-Of( `GET' + SECRET + URI ) ) ) ) )
\end{notice}

Signature Example

\begin{Verbatim}[commandchars=\\\{\}]
base64(sha256(utf8('GET'+'d59b8a775f0dcfd985edff5b5106e2a3'+'/api/v1/patients/1')))
would result in
PHNoYTI1NiBIQVNIIG9iamVjdCBAIDB4MTAzNDhhYTcwPg==
\end{Verbatim}

URL Example

\begin{Verbatim}[commandchars=\\\{\}]
/api/v1/patients/1?signature=PHNoYTI1NiBIQVNIIG9iamVjdCBAIDB4MTAzNDhhYTcwPg==\&timestamp=2014-01-26T00\%3A26\%3A49.000Z\&key=d36a3127fc9f5fa169e911b9ab5b46eb

This assumes:
    Key is d36a3127fc9f5fa169e911b9ab5b46eb
    Secret is d59b8a775f0dcfd985edff5b5106e2a3
    Time is 00:26:49 on 1/26/2014
\end{Verbatim}


\section{Headers}
\label{dev-api-overview:headers}
When interacting with the Cagenix web-service, the following headers are
required to be submitted with each request.

\begin{tabulary}{\linewidth}{|L|L|}
\hline
\textsf{\relax 
Header Element
} & \textsf{\relax 
Description
}\\
\hline
Content-Type
 & 
application/json
Currently XML and other formats are not available.
\\

key
 & 
This is the key provided by the application admins.
\\

datetime
 & 
This is the date and time, ISO format, used in your
header formulas. An example is 2012-10-29T01:30:20Z.
Notice the T and Z, with the T separating the date and
time and Z ending the string.
\\

body\_hash
 & 
This is the encoding of the body contents through the
following formula. If there are no body contents to be
sent, encode and send an empty body array.
\\

signature
 & 
The header signature is a string that combines the body
hash, date and time, along with your secret, encoded to
UTF8 and hashed using sha256.
\\

lang
 & 
The language of the request content. Example is `en'.
\\
\hline\end{tabulary}



\section{Header Formulas}
\label{dev-api-overview:header-formulas}
\begin{tabulary}{\linewidth}{|L|L|}
\hline
\textsf{\relax 
Header Element
} & \textsf{\relax 
Formula
}\\
\hline
datetime
 & 
The date and time is a simple formatting:
yyyy-mm-ddThh:mm:ssZ
\\

body\_hash
 & 
json body content UTF8 encoded, then sha256 hashed.
\\

signature
 & 
body\_hash added to datetime added to secret key, encoded
to UTF8 then sha256 hashed.
\\
\hline\end{tabulary}



\section{Response Codes}
\label{dev-api-overview:response-codes}

\subsection{GET Request Response Codes}
\label{dev-api-overview:get-request-response-codes}
\begin{tabulary}{\linewidth}{|L|L|L|}
\hline
\textsf{\relax 
CODE
} & \textsf{\relax 
Response
} & \textsf{\relax 
Meaning
}\\
\hline
200
 & 
OK
 & 
The request was successful and the body contains
what you asked for.
\\

401
 & 
Unauthorized
 & 
Your credentials do not authorize you to access
the requested info.
\\

404
 & 
Not Found
 & 
The service you requested was not found on our
server.
\\

500
 & 
Server Error
 & 
There was an internal error, if it continues,
please contact us.
\\

503
 & 
Service Unavailable
 & 
The service requested is temporarily down. Try
again later.
\\
\hline\end{tabulary}



\subsection{POST/PUT Request Response Codes}
\label{dev-api-overview:post-put-request-response-codes}
\begin{tabulary}{\linewidth}{|L|L|L|}
\hline
\textsf{\relax 
CODE
} & \textsf{\relax 
Response
} & \textsf{\relax 
Meaning
}\\
\hline
200
 & 
OK
 & 
The request was successful and the body contains
what you asked for.
\\

201
 & 
Created
 & 
The resource you requested by create was
successfully created.
\\

400
 & 
Bad Request
 & 
Something was wrong in your request, check the
message for details.
\\

401
 & 
Unauthorized
 & 
Your credentials do not authorize you to access
the requested info.
\\

404
 & 
Not Found
 & 
The service you requested was not found on our
server.
\\

405
 & 
Method Not Allowed
 & 
You tried POSTing or PUTting to a service that
can’t accept data.
\\

500
 & 
Server Error
 & 
There was an internal error, if it continues,
please contact us.
\\
\hline\end{tabulary}



\subsection{DELETE Request Response Codes}
\label{dev-api-overview:delete-request-response-codes}
\begin{tabulary}{\linewidth}{|L|L|L|}
\hline
\textsf{\relax 
CODE
} & \textsf{\relax 
Response
} & \textsf{\relax 
Meaning
}\\
\hline
204
 & 
OK
 & 
The request was successfully deleted.
\\

400
 & 
Bad Request
 & 
Something was wrong in your request, check the
message for details.
\\

401
 & 
Unauthorized
 & 
Your credentials do not authorize you to access
the requested info.
\\

404
 & 
Not Found
 & 
The service you requested was not found on our
server.
\\

405
 & 
Method Not Allowed
 & 
You tried POSTing or PUTting to a service that
can’t accept data.
\\

500
 & 
Server Error
 & 
There was an internal error, if it continues,
please contact us.
\\
\hline\end{tabulary}



\chapter{Patients API}
\label{dev-api-patients::doc}\label{dev-api-patients:patients-api}

\section{Create a patient}
\label{dev-api-patients:create-a-patient}
This endpoint is used to create a patient in the Cagenix API.

\begin{notice}{note}{Note:}
\textbf{POST} /api/v1/patients/create
\end{notice}


\subsection{Call}
\label{dev-api-patients:call}
\begin{Verbatim}[commandchars=\\\{\}]
\PYG{p}{\PYGZob{}}
    \PYG{l+s+s1}{\PYGZsq{}first\PYGZus{}name\PYGZsq{}}\PYG{o}{:} \PYG{l+s+s1}{\PYGZsq{}\PYGZsq{}}\PYG{p}{,}
    \PYG{l+s+s1}{\PYGZsq{}last\PYGZus{}name\PYGZsq{}}\PYG{o}{:} \PYG{l+s+s1}{\PYGZsq{}\PYGZsq{}}\PYG{p}{,}
    \PYG{l+s+s1}{\PYGZsq{}address\PYGZus{}one\PYGZsq{}}\PYG{o}{:} \PYG{l+s+s1}{\PYGZsq{}\PYGZsq{}}\PYG{p}{,}
    \PYG{l+s+s1}{\PYGZsq{}address\PYGZus{}two\PYGZsq{}}\PYG{o}{:} \PYG{l+s+s1}{\PYGZsq{}\PYGZsq{}}\PYG{p}{,}
    \PYG{l+s+s1}{\PYGZsq{}city\PYGZsq{}}\PYG{o}{:} \PYG{l+s+s1}{\PYGZsq{}\PYGZsq{}}\PYG{p}{,}
    \PYG{l+s+s1}{\PYGZsq{}state\PYGZsq{}}\PYG{o}{:} \PYG{l+s+s1}{\PYGZsq{}\PYGZsq{}}\PYG{p}{,}
    \PYG{l+s+s1}{\PYGZsq{}zip\PYGZus{}code\PYGZsq{}}\PYG{o}{:} \PYG{l+s+s1}{\PYGZsq{}\PYGZsq{}}\PYG{p}{,}
    \PYG{l+s+s1}{\PYGZsq{}email\PYGZsq{}}\PYG{o}{:} \PYG{l+s+s1}{\PYGZsq{}\PYGZsq{}}\PYG{p}{,}
    \PYG{l+s+s1}{\PYGZsq{}practice\PYGZsq{}}\PYG{o}{:} \PYG{l+s+s1}{\PYGZsq{}\PYGZsq{}}\PYG{p}{,}
    \PYG{l+s+s1}{\PYGZsq{}practitioner\PYGZsq{}}\PYG{o}{:} \PYG{l+s+s1}{\PYGZsq{}\PYGZsq{}}\PYG{p}{,}
\PYG{p}{\PYGZcb{}}
\end{Verbatim}

\begin{tabulary}{\linewidth}{|L|L|L|L|}
\hline
\textsf{\relax 
Property
} & \textsf{\relax 
Description
} & \textsf{\relax 
Type
} & \textsf{\relax 
Required
}\\
\hline
first\_name
 & 
First Name
 & 
String
 & 
X
\\

last\_name
 & 
Last Name
 & 
String
 & 
X
\\

address\_one
 & 
First address line
 & 
String
 & 
X
\\

address\_two
 & 
Second address line
 & 
String
 & \\

city
 & 
City
 & 
String
 & 
X
\\

state
 & 
State
 & 
String
 & 
X
\\

zip\_code
 & 
ZIP Code in 5-4 format
 & 
String
 & 
X
\\

email
 & 
email address
 & 
String
 & 
X
\\

practice
 & 
ID of the Practice
 & 
Integer
 & \\

practitioner
 & 
ID of the Practitioner
 & 
Integer
 & 
X
\\
\hline\end{tabulary}



\subsection{Response}
\label{dev-api-patients:response}
\begin{Verbatim}[commandchars=\\\{\}]
\PYG{p}{\PYGZob{}}
    \PYG{l+s+s1}{\PYGZsq{}request\PYGZsq{}}\PYG{o}{:} \PYG{p}{\PYGZob{}}
        \PYG{l+s+s1}{\PYGZsq{}first\PYGZus{}name\PYGZsq{}}\PYG{o}{:} \PYG{l+s+s1}{\PYGZsq{}\PYGZsq{}}\PYG{p}{,}
        \PYG{l+s+s1}{\PYGZsq{}last\PYGZus{}name\PYGZsq{}}\PYG{o}{:} \PYG{l+s+s1}{\PYGZsq{}\PYGZsq{}}\PYG{p}{,}
        \PYG{l+s+s1}{\PYGZsq{}address\PYGZus{}one\PYGZsq{}}\PYG{o}{:} \PYG{l+s+s1}{\PYGZsq{}\PYGZsq{}}\PYG{p}{,}
        \PYG{l+s+s1}{\PYGZsq{}address\PYGZus{}two\PYGZsq{}}\PYG{o}{:} \PYG{l+s+s1}{\PYGZsq{}\PYGZsq{}}\PYG{p}{,}
        \PYG{l+s+s1}{\PYGZsq{}city\PYGZsq{}}\PYG{o}{:} \PYG{l+s+s1}{\PYGZsq{}\PYGZsq{}}\PYG{p}{,}
        \PYG{l+s+s1}{\PYGZsq{}state\PYGZsq{}}\PYG{o}{:} \PYG{l+s+s1}{\PYGZsq{}\PYGZsq{}}\PYG{p}{,}
        \PYG{l+s+s1}{\PYGZsq{}zip\PYGZus{}code\PYGZsq{}}\PYG{o}{:} \PYG{l+s+s1}{\PYGZsq{}\PYGZsq{}}\PYG{p}{,}
        \PYG{l+s+s1}{\PYGZsq{}email\PYGZsq{}}\PYG{o}{:} \PYG{l+s+s1}{\PYGZsq{}\PYGZsq{}}\PYG{p}{,}
        \PYG{l+s+s1}{\PYGZsq{}practice\PYGZsq{}}\PYG{o}{:} \PYG{l+s+s1}{\PYGZsq{}\PYGZsq{}}\PYG{p}{,}
        \PYG{l+s+s1}{\PYGZsq{}practitioner\PYGZsq{}}\PYG{o}{:} \PYG{l+s+s1}{\PYGZsq{}\PYGZsq{}}\PYG{p}{,}
    \PYG{p}{\PYGZcb{}}\PYG{p}{,}
    \PYG{l+s+s1}{\PYGZsq{}response\PYGZsq{}}\PYG{o}{:} \PYG{p}{\PYGZob{}}
        \PYG{l+s+s1}{\PYGZsq{}patient\PYGZus{}id\PYGZsq{}}\PYG{o}{:} \PYG{l+s+s1}{\PYGZsq{}\PYGZsq{}}\PYG{p}{,}
    \PYG{p}{\PYGZcb{}}\PYG{p}{,}
\PYG{p}{\PYGZcb{}}
\end{Verbatim}

\begin{tabulary}{\linewidth}{|L|L|L|}
\hline
\textsf{\relax 
Property
} & \textsf{\relax 
Description
} & \textsf{\relax 
Type
}\\
\hline
request
 & 
Object containing the original
request data
 & 
Object
\\

response
 & 
Object containing the Patient ID
 & 
Object
\\

patient\_id
 & 
A unique ID for the Patient record
 & 
Integer
\\
\hline\end{tabulary}



\section{Retrieve Patient details}
\label{dev-api-patients:retrieve-patient-details}
This endpoint is used to retrieve all the details of a Patient.

\begin{notice}{note}{Note:}
\textbf{GET} /api/v1/patients/\textless{}ID\textgreater{}
\end{notice}


\subsection{Response}
\label{dev-api-patients:id1}
\begin{Verbatim}[commandchars=\\\{\}]
\PYG{p}{\PYGZob{}}
    \PYG{l+s+s1}{\PYGZsq{}request\PYGZsq{}}\PYG{o}{:} \PYG{p}{\PYGZob{}}
        \PYG{l+s+s1}{\PYGZsq{}patient\PYGZus{}id\PYGZsq{}}\PYG{o}{:} \PYG{l+s+s1}{\PYGZsq{}\PYGZsq{}}\PYG{p}{,}
    \PYG{p}{\PYGZcb{}}\PYG{p}{,}
    \PYG{l+s+s1}{\PYGZsq{}response\PYGZsq{}}\PYG{o}{:} \PYG{p}{\PYGZob{}}
        \PYG{l+s+s1}{\PYGZsq{}active\PYGZsq{}}\PYG{o}{:} \PYG{l+s+s1}{\PYGZsq{}\PYGZsq{}}\PYG{p}{,}
        \PYG{l+s+s1}{\PYGZsq{}first\PYGZus{}name\PYGZsq{}}\PYG{o}{:} \PYG{l+s+s1}{\PYGZsq{}\PYGZsq{}}\PYG{p}{,}
        \PYG{l+s+s1}{\PYGZsq{}last\PYGZus{}name\PYGZsq{}}\PYG{o}{:} \PYG{l+s+s1}{\PYGZsq{}\PYGZsq{}}\PYG{p}{,}
        \PYG{l+s+s1}{\PYGZsq{}address\PYGZus{}one\PYGZsq{}}\PYG{o}{:} \PYG{l+s+s1}{\PYGZsq{}\PYGZsq{}}\PYG{p}{,}
        \PYG{l+s+s1}{\PYGZsq{}address\PYGZus{}two\PYGZsq{}}\PYG{o}{:} \PYG{l+s+s1}{\PYGZsq{}\PYGZsq{}}\PYG{p}{,}
        \PYG{l+s+s1}{\PYGZsq{}city\PYGZsq{}}\PYG{o}{:} \PYG{l+s+s1}{\PYGZsq{}\PYGZsq{}}\PYG{p}{,}
        \PYG{l+s+s1}{\PYGZsq{}state\PYGZsq{}}\PYG{o}{:} \PYG{l+s+s1}{\PYGZsq{}\PYGZsq{}}\PYG{p}{,}
        \PYG{l+s+s1}{\PYGZsq{}zip\PYGZus{}code\PYGZsq{}}\PYG{o}{:} \PYG{l+s+s1}{\PYGZsq{}\PYGZsq{}}\PYG{p}{,}
        \PYG{l+s+s1}{\PYGZsq{}email\PYGZsq{}}\PYG{o}{:} \PYG{l+s+s1}{\PYGZsq{}\PYGZsq{}}\PYG{p}{,}
        \PYG{l+s+s1}{\PYGZsq{}choice\PYGZus{}one\PYGZsq{}}\PYG{o}{:} \PYG{l+s+s1}{\PYGZsq{}\PYGZsq{}}\PYG{p}{,}
        \PYG{l+s+s1}{\PYGZsq{}choice\PYGZus{}two\PYGZsq{}}\PYG{o}{:} \PYG{l+s+s1}{\PYGZsq{}\PYGZsq{}}\PYG{p}{,}
        \PYG{l+s+s1}{\PYGZsq{}choice\PYGZus{}three\PYGZsq{}}\PYG{o}{:} \PYG{l+s+s1}{\PYGZsq{}\PYGZsq{}}\PYG{p}{,}
        \PYG{l+s+s1}{\PYGZsq{}practice\PYGZsq{}}\PYG{o}{:} \PYG{l+s+s1}{\PYGZsq{}\PYGZsq{}}\PYG{p}{,}
        \PYG{l+s+s1}{\PYGZsq{}practitioner\PYGZsq{}}\PYG{o}{:} \PYG{l+s+s1}{\PYGZsq{}\PYGZsq{}}\PYG{p}{,}
    \PYG{p}{\PYGZcb{}}
\PYG{p}{\PYGZcb{}}
\end{Verbatim}

\begin{tabulary}{\linewidth}{|L|L|L|}
\hline
\textsf{\relax 
Property
} & \textsf{\relax 
Description
} & \textsf{\relax 
Type
}\\
\hline
active
 & 
Active Flag for Archiving
 & 
String
\\

first\_name
 & 
First Name
 & 
String
\\

last\_name
 & 
Last Name
 & 
String
\\

address\_one
 & 
First address line
 & 
String
\\

address\_two
 & 
Second address line
 & 
String
\\

city
 & 
City
 & 
String
\\

state
 & 
State
 & 
String
\\

zip\_code
 & 
ZIP Code in 5-4 format
 & 
String
\\

email
 & 
email address
 & 
String
\\

choice\_one
 & 
The ``best'' choice
 & 
String
\\

choice\_two
 & 
The ``better'' choice
 & 
String
\\

choice\_three
 & 
The ``good'' choice
 & 
String
\\

practice
 & 
ID of the Practice
 & 
Integer
\\

practitioner
 & 
ID of the Practitioner
 & 
Integer
\\
\hline\end{tabulary}



\section{Update a patient}
\label{dev-api-patients:update-a-patient}
This endpoint is used to update a patient in the Cagenix API.

\begin{notice}{note}{Note:}
\textbf{PUT} /api/v1/patients/\textless{}id\textgreater{}
\end{notice}


\subsection{Call}
\label{dev-api-patients:id2}
\begin{Verbatim}[commandchars=\\\{\}]
\PYG{p}{\PYGZob{}}
    \PYG{l+s+s1}{\PYGZsq{}first\PYGZus{}name\PYGZsq{}}\PYG{o}{:} \PYG{l+s+s1}{\PYGZsq{}\PYGZsq{}}\PYG{p}{,}
    \PYG{l+s+s1}{\PYGZsq{}last\PYGZus{}name\PYGZsq{}}\PYG{o}{:} \PYG{l+s+s1}{\PYGZsq{}\PYGZsq{}}\PYG{p}{,}
    \PYG{l+s+s1}{\PYGZsq{}address\PYGZus{}one\PYGZsq{}}\PYG{o}{:} \PYG{l+s+s1}{\PYGZsq{}\PYGZsq{}}\PYG{p}{,}
    \PYG{l+s+s1}{\PYGZsq{}address\PYGZus{}two\PYGZsq{}}\PYG{o}{:} \PYG{l+s+s1}{\PYGZsq{}\PYGZsq{}}\PYG{p}{,}
    \PYG{l+s+s1}{\PYGZsq{}city\PYGZsq{}}\PYG{o}{:} \PYG{l+s+s1}{\PYGZsq{}\PYGZsq{}}\PYG{p}{,}
    \PYG{l+s+s1}{\PYGZsq{}state\PYGZsq{}}\PYG{o}{:} \PYG{l+s+s1}{\PYGZsq{}\PYGZsq{}}\PYG{p}{,}
    \PYG{l+s+s1}{\PYGZsq{}zip\PYGZus{}code\PYGZsq{}}\PYG{o}{:} \PYG{l+s+s1}{\PYGZsq{}\PYGZsq{}}\PYG{p}{,}
    \PYG{l+s+s1}{\PYGZsq{}email\PYGZsq{}}\PYG{o}{:} \PYG{l+s+s1}{\PYGZsq{}\PYGZsq{}}\PYG{p}{,}
    \PYG{l+s+s1}{\PYGZsq{}active\PYGZsq{}}\PYG{o}{:} \PYG{l+s+s1}{\PYGZsq{}\PYGZsq{}}\PYG{p}{,}
    \PYG{l+s+s1}{\PYGZsq{}practice\PYGZsq{}}\PYG{o}{:} \PYG{l+s+s1}{\PYGZsq{}\PYGZsq{}}\PYG{p}{,}
    \PYG{l+s+s1}{\PYGZsq{}practitioner\PYGZsq{}}\PYG{o}{:} \PYG{l+s+s1}{\PYGZsq{}\PYGZsq{}}\PYG{p}{,}
\PYG{p}{\PYGZcb{}}
\end{Verbatim}

\begin{tabulary}{\linewidth}{|L|L|L|L|}
\hline
\textsf{\relax 
Property
} & \textsf{\relax 
Description
} & \textsf{\relax 
Type
} & \textsf{\relax 
Required
}\\
\hline
active
 & 
Active Flag for Archiving
 & 
Boolean
 & \\

patient\_id
 & 
Patient ID
 & 
String
 & 
X
\\

first\_name
 & 
First Name
 & 
String
 & \\

last\_name
 & 
Last Name
 & 
String
 & \\

address\_one
 & 
First address line
 & 
String
 & \\

address\_two
 & 
Second address line
 & 
String
 & \\

city
 & 
City
 & 
String
 & \\

state
 & 
State
 & 
String
 & \\

zip\_code
 & 
ZIP Code in 5-4 format
 & 
String
 & \\

email
 & 
email address
 & 
String
 & \\

active
 & 
Active Flag for Archiving
 & 
Boolean
 & \\

practice
 & 
ID of the Practice
 & 
Integer
 & \\

practitioner
 & 
ID of the Practitioner
 & 
Integer
 & \\
\hline\end{tabulary}



\subsection{Response}
\label{dev-api-patients:id3}
\begin{Verbatim}[commandchars=\\\{\}]
\PYG{p}{\PYGZob{}}
    \PYG{l+s+s1}{\PYGZsq{}request\PYGZsq{}}\PYG{o}{:} \PYG{p}{\PYGZob{}}
        \PYG{l+s+s1}{\PYGZsq{}patient\PYGZus{}id\PYGZsq{}}\PYG{o}{:} \PYG{l+s+s1}{\PYGZsq{}\PYGZsq{}}\PYG{p}{,}
        \PYG{l+s+s1}{\PYGZsq{}first\PYGZus{}name\PYGZsq{}}\PYG{o}{:} \PYG{l+s+s1}{\PYGZsq{}\PYGZsq{}}\PYG{p}{,}
        \PYG{l+s+s1}{\PYGZsq{}last\PYGZus{}name\PYGZsq{}}\PYG{o}{:} \PYG{l+s+s1}{\PYGZsq{}\PYGZsq{}}\PYG{p}{,}
        \PYG{l+s+s1}{\PYGZsq{}address\PYGZus{}one\PYGZsq{}}\PYG{o}{:} \PYG{l+s+s1}{\PYGZsq{}\PYGZsq{}}\PYG{p}{,}
        \PYG{l+s+s1}{\PYGZsq{}address\PYGZus{}two\PYGZsq{}}\PYG{o}{:} \PYG{l+s+s1}{\PYGZsq{}\PYGZsq{}}\PYG{p}{,}
        \PYG{l+s+s1}{\PYGZsq{}city\PYGZsq{}}\PYG{o}{:} \PYG{l+s+s1}{\PYGZsq{}\PYGZsq{}}\PYG{p}{,}
        \PYG{l+s+s1}{\PYGZsq{}state\PYGZsq{}}\PYG{o}{:} \PYG{l+s+s1}{\PYGZsq{}\PYGZsq{}}\PYG{p}{,}
        \PYG{l+s+s1}{\PYGZsq{}zip\PYGZus{}code\PYGZsq{}}\PYG{o}{:} \PYG{l+s+s1}{\PYGZsq{}\PYGZsq{}}\PYG{p}{,}
        \PYG{l+s+s1}{\PYGZsq{}email\PYGZsq{}}\PYG{o}{:} \PYG{l+s+s1}{\PYGZsq{}\PYGZsq{}}\PYG{p}{,}
        \PYG{l+s+s1}{\PYGZsq{}active\PYGZsq{}}\PYG{o}{:} \PYG{l+s+s1}{\PYGZsq{}\PYGZsq{}}\PYG{p}{,}
        \PYG{l+s+s1}{\PYGZsq{}practice\PYGZsq{}}\PYG{o}{:} \PYG{l+s+s1}{\PYGZsq{}\PYGZsq{}}\PYG{p}{,}
        \PYG{l+s+s1}{\PYGZsq{}practitioner\PYGZsq{}}\PYG{o}{:} \PYG{l+s+s1}{\PYGZsq{}\PYGZsq{}}\PYG{p}{,}
    \PYG{p}{\PYGZcb{}}\PYG{p}{,}
    \PYG{l+s+s1}{\PYGZsq{}response\PYGZsq{}}\PYG{o}{:} \PYG{p}{\PYGZob{}}
        \PYG{l+s+s1}{\PYGZsq{}patient\PYGZus{}id\PYGZsq{}}\PYG{o}{:} \PYG{l+s+s1}{\PYGZsq{}\PYGZsq{}}\PYG{p}{,}
        \PYG{l+s+s1}{\PYGZsq{}active\PYGZsq{}}\PYG{o}{:} \PYG{l+s+s1}{\PYGZsq{}\PYGZsq{}}\PYG{p}{,}
    \PYG{p}{\PYGZcb{}}\PYG{p}{,}
\PYG{p}{\PYGZcb{}}
\end{Verbatim}

\begin{tabulary}{\linewidth}{|L|L|L|}
\hline
\textsf{\relax 
Property
} & \textsf{\relax 
Description
} & \textsf{\relax 
Type
}\\
\hline
request
 & 
Object containing the original
request data
 & 
Object
\\

response
 & 
Object containing the Patient ID
 & 
Object
\\

patient\_id
 & 
A unique ID for the Patient record
 & 
Integer
\\

active
 & 
Active Flag for Archiving
 & 
Boolean
\\
\hline\end{tabulary}



\section{Delete a patient}
\label{dev-api-patients:delete-a-patient}
This endpoint is used to Delete a patient in the Cagenix API.

\begin{notice}{note}{Note:}
\textbf{DELETE} /api/v1/patients/\textless{}id\textgreater{}
\end{notice}


\subsection{Response}
\label{dev-api-patients:id4}
\begin{Verbatim}[commandchars=\\\{\}]
\PYG{p}{\PYGZob{}}
    \PYG{l+s+s1}{\PYGZsq{}request\PYGZsq{}}\PYG{o}{:} \PYG{p}{\PYGZob{}}
        \PYG{l+s+s1}{\PYGZsq{}patient\PYGZus{}id\PYGZsq{}}\PYG{o}{:} \PYG{l+s+s1}{\PYGZsq{}\PYGZsq{}}\PYG{p}{,}
    \PYG{p}{\PYGZcb{}}\PYG{p}{,}
    \PYG{l+s+s1}{\PYGZsq{}response\PYGZsq{}}\PYG{o}{:} \PYG{p}{\PYGZob{}}
        \PYG{l+s+s1}{\PYGZsq{}status\PYGZsq{}}\PYG{o}{:} \PYG{l+s+s1}{\PYGZsq{}\PYGZsq{}}\PYG{p}{,}
    \PYG{p}{\PYGZcb{}}\PYG{p}{,}
\PYG{p}{\PYGZcb{}}
\end{Verbatim}

\begin{tabulary}{\linewidth}{|L|L|L|}
\hline
\textsf{\relax 
Property
} & \textsf{\relax 
Description
} & \textsf{\relax 
Type
}\\
\hline
request
 & 
Object containing the original
request data
 & 
Object
\\

response
 & 
Object containing the Patient ID
 & 
Object
\\

status
 & 
The result of the DELETE
opperation (EX: Success, Failed)
 & 
String
\\
\hline\end{tabulary}



\chapter{Evaluations API}
\label{dev-api-evaluations::doc}\label{dev-api-evaluations:evaluations-api}

\section{Create an Evaluation}
\label{dev-api-evaluations:create-an-evaluation}
This endpoint is used to create an evaluation in the Cagenix API.

\begin{notice}{note}{Note:}
\textbf{POST} /api/v1/evaluations/create
\end{notice}


\subsection{Call}
\label{dev-api-evaluations:call}
\begin{Verbatim}[commandchars=\\\{\}]
\PYGZob{}
    'practice': '',
    'practitioner': '',
    'patient': '',
    'answers': [
        \PYGZob{}
            'question\_id': '',
            'patient\_answer': '',
        \PYGZcb{}, \#...
    ]
\PYGZcb{}
\end{Verbatim}

\begin{tabulary}{\linewidth}{|L|L|L|L|}
\hline
\textsf{\relax 
Property
} & \textsf{\relax 
Description
} & \textsf{\relax 
Type
} & \textsf{\relax 
Required
}\\
\hline
practice
 & 
ID of the Practice
 & 
Integer
 & \\

practitioner
 & 
ID of the Practitioner
 & 
Integer
 & \\

patient
 & 
ID of the Patient
 & 
Integer
 & 
X
\\

answers
 & 
A collection of Answers objects
 & 
List
 & 
X
\\

question\_id
 & 
ID of the Evaluation Question
 & 
Integer
 & 
X
\\

patient\_answer
 & 
Patient's response to the question
 & 
String
 & 
X
\\
\hline\end{tabulary}



\subsection{Response}
\label{dev-api-evaluations:response}
\begin{Verbatim}[commandchars=\\\{\}]
\PYGZob{}
    'request': \PYGZob{}
        'practice': '',
        'practitioner': '',
        'patient': '',
        'answers': [
            \PYGZob{}
                'question\_id': '',
                'patient\_answer': '',
            \PYGZcb{}, \#...
        ]
    \PYGZcb{},
    'response': \PYGZob{}
        'evaluation\_uuid': '',
    \PYGZcb{},
\PYGZcb{}
\end{Verbatim}

\begin{tabulary}{\linewidth}{|L|L|L|}
\hline
\textsf{\relax 
Property
} & \textsf{\relax 
Description
} & \textsf{\relax 
Type
}\\
\hline
request
 & 
Object containing the original
request data
 & 
Object
\\

response
 & 
Object containing the Eval ID
 & 
Object
\\

evaluation\_uuid
 & 
A unique UUID for the evaulation
records
 & 
String
\\
\hline\end{tabulary}



\section{Retrieve Evaluation details}
\label{dev-api-evaluations:retrieve-evaluation-details}
This endpoint is used to retrieve all the details of an Evaluation.

\begin{notice}{note}{Note:}
\textbf{GET} /api/v1/evaluations/\textless{}UUID\textgreater{}
\end{notice}


\subsection{Response}
\label{dev-api-evaluations:id1}
\begin{Verbatim}[commandchars=\\\{\}]
\PYGZob{}
    'request': \PYGZob{}
        'evaluation\_uuid': '',
    \PYGZcb{},
    'response': \PYGZob{}
        'evaluation\_uuid': '',
        'practice': '',
        'practitioner': '',
        'patient': '',
        'answers': [
            \PYGZob{}
                'question\_id': '',
                'patient\_answer': '',
            \PYGZcb{}, \#...
        ]
        'active': '',
    \PYGZcb{}
\PYGZcb{}
\end{Verbatim}

\begin{tabulary}{\linewidth}{|L|L|L|}
\hline
\textsf{\relax 
Property
} & \textsf{\relax 
Description
} & \textsf{\relax 
Type
}\\
\hline
evaluation\_uuid
 & 
A unique UUID for the evaulation
records
 & 
String
\\

practice
 & 
ID of the Practice
 & 
Integer
\\

practitioner
 & 
ID of the Practitioner
 & 
Integer
\\

patient
 & 
ID of the Patient
 & 
Integer
\\

answers
 & 
A collection of Answers objects
 & 
List
\\

question\_id
 & 
ID of the Evaluation Question
 & 
Integer
\\

patient\_answer
 & 
Patient's response to the question
 & 
String
\\
\hline\end{tabulary}



\section{Update an Evaluation}
\label{dev-api-evaluations:update-an-evaluation}
This endpoint is used to update an evaluation in the Cagenix API. You must
resubmit all answer objects again.  So if an evaulation had 20 questions, all of
those answers must be resubmitted for an update.

\begin{notice}{note}{Note:}
\textbf{PUT} /api/v1/evaluations/\textless{}uuid\textgreater{}
\end{notice}


\subsection{Call}
\label{dev-api-evaluations:id2}
\begin{Verbatim}[commandchars=\\\{\}]
\PYGZob{}
    'evaluation\_uuid': '',
    'practice': '',
    'practitioner': '',
    'patient': '',
    'answers': [
        \PYGZob{}
            'question\_id': '',
            'patient\_answer': '',
        \PYGZcb{}, \#...
    ]
    'active': '',
\PYGZcb{}
\end{Verbatim}

\begin{tabulary}{\linewidth}{|L|L|L|L|}
\hline
\textsf{\relax 
Property
} & \textsf{\relax 
Description
} & \textsf{\relax 
Type
} & \textsf{\relax 
Required
}\\
\hline
evaluation\_uuid
 & 
A unique UUID for the evaulation
records
 & 
String
 & 
X
\\

practice
 & 
ID of the Practice
 & 
Integer
 & \\

practitioner
 & 
ID of the Practitioner
 & 
Integer
 & \\

patient
 & 
ID of the Patient
 & 
Integer
 & \\

answers
 & 
A collection of Answers objects
 & 
List
 & 
X
\\

question\_id
 & 
ID of the Evaluation Question
 & 
Integer
 & 
X
\\

patient\_answer
 & 
Patient's response to the question
 & 
String
 & 
X
\\
\hline\end{tabulary}



\subsection{Response}
\label{dev-api-evaluations:id3}
\begin{Verbatim}[commandchars=\\\{\}]
\PYGZob{}
    'request': \PYGZob{}
        'evaluation\_uuid': '',
        'practice': '',
        'practitioner': '',
        'patient': '',
        'answers': [
            \PYGZob{}
                'question\_id': '',
                'patient\_answer': '',
            \PYGZcb{}, \#...
        ]
        'active': '',
    \PYGZcb{},
    'response': \PYGZob{}
        'evaluation\_uuid': '',
        'status': '',
        'active': '',
    \PYGZcb{},
\PYGZcb{}
\end{Verbatim}

\begin{tabulary}{\linewidth}{|L|L|L|}
\hline
\textsf{\relax 
Property
} & \textsf{\relax 
Description
} & \textsf{\relax 
Type
}\\
\hline
request
 & 
Object containing the original
request data
 & 
Object
\\

response
 & 
Object containing the Patient ID
 & 
Object
\\

evaluation\_uuid
 & 
A unique UUID for the evaulation
records
 & 
String
\\

status
 & 
The result of the PUT opperation
(EX: Success, Failed)
 & 
String
\\

active
 & 
Active Flag for Archiving
 & 
Boolean
\\
\hline\end{tabulary}



\section{Delete a patient}
\label{dev-api-evaluations:delete-a-patient}
This endpoint is used to delete an evaluation in the Cagenix API.

\begin{notice}{note}{Note:}
\textbf{DELETE} /api/v1/evaluations/\textless{}uuid\textgreater{}
\end{notice}


\subsection{Response}
\label{dev-api-evaluations:id4}
\begin{Verbatim}[commandchars=\\\{\}]
\PYG{p}{\PYGZob{}}
    \PYG{l+s+s1}{\PYGZsq{}request\PYGZsq{}}\PYG{o}{:} \PYG{p}{\PYGZob{}}
        \PYG{l+s+s1}{\PYGZsq{}evaluation\PYGZus{}uuid\PYGZsq{}}\PYG{o}{:} \PYG{l+s+s1}{\PYGZsq{}\PYGZsq{}}\PYG{p}{,}
    \PYG{p}{\PYGZcb{}}\PYG{p}{,}
    \PYG{l+s+s1}{\PYGZsq{}response\PYGZsq{}}\PYG{o}{:} \PYG{p}{\PYGZob{}}
        \PYG{l+s+s1}{\PYGZsq{}status\PYGZsq{}}\PYG{o}{:} \PYG{l+s+s1}{\PYGZsq{}\PYGZsq{}}\PYG{p}{,}
    \PYG{p}{\PYGZcb{}}\PYG{p}{,}
\PYG{p}{\PYGZcb{}}
\end{Verbatim}

\begin{tabulary}{\linewidth}{|L|L|L|}
\hline
\textsf{\relax 
Property
} & \textsf{\relax 
Description
} & \textsf{\relax 
Type
}\\
\hline
request
 & 
Object containing the original
request data
 & 
Object
\\

response
 & 
Object containing the Evaluation
UUID
 & 
Object
\\

status
 & 
The result of the DELETE
opperation (EX: Success, Failed)
 & 
String
\\
\hline\end{tabulary}



\chapter{Users Blueprint Overview}
\label{dev-users::doc}\label{dev-users:users-blueprint-overview}
The following document covers the internals of the Cagenix-Web Users
Blueprint.
\begin{itemize}
\item {} 
{\hyperref[dev-users:users-models-label]{\emph{Users Models}}}

\item {} 
\emph{users-forms-label}

\item {} 
\emph{users-views-label}

\end{itemize}


\section{Users Models}
\label{dev-users:users-models-label}\label{dev-users:module-cagenix.users.models}\label{dev-users:users-models}\index{cagenix.users.models (module)}
cagenix.users.models

This file is used to define all the models used by the Users blueprint
\index{Role (class in cagenix.users.models)}

\begin{fulllineitems}
\phantomsection\label{dev-users:cagenix.users.models.Role}\pysiglinewithargsret{\strong{class }\code{cagenix.users.models.}\bfcode{Role}}{\emph{name}, \emph{description=None}}{}
The Role model defines the authorization roles used by Flask-Security

This Role class defines the model attributes, and several classmethods to
make using the class easier.
\begin{quote}\begin{description}
\item[{Parameters}] \leavevmode\begin{itemize}
\item {} 
\textbf{db.Model} (\emph{object}) -- The SQLAlchemy base database model

\item {} 
\textbf{RoleMixin} (\emph{object}) -- The RoleMixin base provided by Flask-Security

\end{itemize}

\end{description}\end{quote}
\index{by\_id() (cagenix.users.models.Role class method)}

\begin{fulllineitems}
\phantomsection\label{dev-users:cagenix.users.models.Role.by_id}\pysiglinewithargsret{\strong{classmethod }\bfcode{by\_id}}{\emph{role\_id}}{}
by\_id returns a Role object given a role\_id
\begin{quote}\begin{description}
\item[{Parameters}] \leavevmode\begin{itemize}
\item {} 
\textbf{cls} (\emph{class}) -- The calling class

\item {} 
\textbf{role\_id} (\emph{int}) -- The id of the role being searched for

\end{itemize}

\end{description}\end{quote}

\end{fulllineitems}

\index{by\_name() (cagenix.users.models.Role class method)}

\begin{fulllineitems}
\phantomsection\label{dev-users:cagenix.users.models.Role.by_name}\pysiglinewithargsret{\strong{classmethod }\bfcode{by\_name}}{\emph{role\_name}}{}
by\_name returns a Role object given a role\_name
\begin{quote}\begin{description}
\item[{Parameters}] \leavevmode\begin{itemize}
\item {} 
\textbf{cls} (\emph{class}) -- The calling class

\item {} 
\textbf{role\_name} (\emph{string}) -- The name of the role being searched for

\end{itemize}

\end{description}\end{quote}

\end{fulllineitems}

\index{get\_roles() (cagenix.users.models.Role class method)}

\begin{fulllineitems}
\phantomsection\label{dev-users:cagenix.users.models.Role.get_roles}\pysiglinewithargsret{\strong{classmethod }\bfcode{get\_roles}}{}{}
get\_roles returns a list of Role objects

This is primarily used by the WTForms to display the QueryMultipleSelect
\begin{quote}\begin{description}
\item[{Parameters}] \leavevmode
\textbf{cls} (\emph{class}) -- The calling class

\end{description}\end{quote}

\end{fulllineitems}


\end{fulllineitems}

\index{User (class in cagenix.users.models)}

\begin{fulllineitems}
\phantomsection\label{dev-users:cagenix.users.models.User}\pysiglinewithargsret{\strong{class }\code{cagenix.users.models.}\bfcode{User}}{\emph{**kwargs}}{}
The User model defines the authentication users used by Flask-Security

This Role class defines the model attributes, a classmethod to make finding
the users easier. It also contains several fields unexposed by the UI. Those
fields are used to track user logins and access IPs.
\begin{quote}\begin{description}
\item[{Parameters}] \leavevmode\begin{itemize}
\item {} 
\textbf{db.Model} (\emph{object}) -- The SQLAlchemy base database model

\item {} 
\textbf{UserMixin} (\emph{object}) -- The UserMixin base provided by Flask-Security

\end{itemize}

\end{description}\end{quote}
\index{by\_email() (cagenix.users.models.User class method)}

\begin{fulllineitems}
\phantomsection\label{dev-users:cagenix.users.models.User.by_email}\pysiglinewithargsret{\strong{classmethod }\bfcode{by\_email}}{\emph{email}}{}
by\_email returns a User object given an email address.
\begin{quote}\begin{description}
\item[{Parameters}] \leavevmode\begin{itemize}
\item {} 
\textbf{cls} (\emph{class}) -- The calling class

\item {} 
\textbf{email} (\emph{string}) -- The email of the user being searched for

\end{itemize}

\end{description}\end{quote}

\end{fulllineitems}

\index{by\_id() (cagenix.users.models.User class method)}

\begin{fulllineitems}
\phantomsection\label{dev-users:cagenix.users.models.User.by_id}\pysiglinewithargsret{\strong{classmethod }\bfcode{by\_id}}{\emph{user\_id}}{}
by\_id returns a User object given a user\_id
\begin{quote}\begin{description}
\item[{Parameters}] \leavevmode\begin{itemize}
\item {} 
\textbf{cls} (\emph{class}) -- The calling class

\item {} 
\textbf{user\_id} (\emph{int}) -- The id of the user being searched for

\end{itemize}

\end{description}\end{quote}

\end{fulllineitems}

\index{by\_phone() (cagenix.users.models.User class method)}

\begin{fulllineitems}
\phantomsection\label{dev-users:cagenix.users.models.User.by_phone}\pysiglinewithargsret{\strong{classmethod }\bfcode{by\_phone}}{\emph{phone}}{}
by\_phone returns a User object given a phone number
\begin{quote}\begin{description}
\item[{Parameters}] \leavevmode\begin{itemize}
\item {} 
\textbf{cls} (\emph{class}) -- The calling class

\item {} 
\textbf{phone} (\emph{string}) -- The phone number of the user being searched for

\end{itemize}

\end{description}\end{quote}

\end{fulllineitems}

\index{by\_username() (cagenix.users.models.User class method)}

\begin{fulllineitems}
\phantomsection\label{dev-users:cagenix.users.models.User.by_username}\pysiglinewithargsret{\strong{classmethod }\bfcode{by\_username}}{\emph{username}}{}
by\_username returns a User object given a username
\begin{quote}\begin{description}
\item[{Parameters}] \leavevmode\begin{itemize}
\item {} 
\textbf{cls} (\emph{class}) -- The calling class

\item {} 
\textbf{username} (\emph{string}) -- The username of the user being searched for

\end{itemize}

\end{description}\end{quote}

\end{fulllineitems}


\end{fulllineitems}



\chapter{Patients Blueprint Overview}
\label{dev-patients::doc}\label{dev-patients:patients-blueprint-overview}
The following document covers the internals of the Cagenix-Web Patients
Blueprint.
\begin{itemize}
\item {} 
{\hyperref[dev-patients:patients-models-label]{\emph{Patients Models}}}

\item {} 
\emph{patients-forms-label}

\item {} 
\emph{patients-views-label}

\end{itemize}


\section{Patients Models}
\label{dev-patients:patients-models-label}\label{dev-patients:patients-models}\label{dev-patients:module-cagenix.patients.models}\index{cagenix.patients.models (module)}
cagenix.users.models

This file is used to define all the models used by the Users blueprint
\index{Patient (class in cagenix.patients.models)}

\begin{fulllineitems}
\phantomsection\label{dev-patients:cagenix.patients.models.Patient}\pysiglinewithargsret{\strong{class }\code{cagenix.patients.models.}\bfcode{Patient}}{\emph{first\_name=None}, \emph{last\_name=None}}{}
The User model defines the authentication users used by Flask-Security

This Role class defines the model attributes, a classmethod to make finding
the users easier. It also contains several fields unexposed by the UI. Those
fields are used to track user logins and access IPs.
\begin{quote}\begin{description}
\item[{Parameters}] \leavevmode\begin{itemize}
\item {} 
\textbf{db.Model} (\emph{object}) -- The SQLAlchemy base database model

\item {} 
\textbf{UserMixin} (\emph{object}) -- The UserMixin base provided by Flask-Security

\end{itemize}

\end{description}\end{quote}
\index{by\_email() (cagenix.patients.models.Patient class method)}

\begin{fulllineitems}
\phantomsection\label{dev-patients:cagenix.patients.models.Patient.by_email}\pysiglinewithargsret{\strong{classmethod }\bfcode{by\_email}}{\emph{email}}{}
by\_email returns a User object given an email address.
\begin{quote}\begin{description}
\item[{Parameters}] \leavevmode\begin{itemize}
\item {} 
\textbf{cls} (\emph{class}) -- The calling class

\item {} 
\textbf{email} (\emph{string}) -- The email of the user being searched for

\end{itemize}

\end{description}\end{quote}

\end{fulllineitems}

\index{by\_id() (cagenix.patients.models.Patient class method)}

\begin{fulllineitems}
\phantomsection\label{dev-patients:cagenix.patients.models.Patient.by_id}\pysiglinewithargsret{\strong{classmethod }\bfcode{by\_id}}{\emph{user\_id}}{}
by\_id returns a User object given a user\_id
\begin{quote}\begin{description}
\item[{Parameters}] \leavevmode\begin{itemize}
\item {} 
\textbf{cls} (\emph{class}) -- The calling class

\item {} 
\textbf{user\_id} (\emph{int}) -- The id of the user being searched for

\end{itemize}

\end{description}\end{quote}

\end{fulllineitems}

\index{by\_phone() (cagenix.patients.models.Patient class method)}

\begin{fulllineitems}
\phantomsection\label{dev-patients:cagenix.patients.models.Patient.by_phone}\pysiglinewithargsret{\strong{classmethod }\bfcode{by\_phone}}{\emph{phone}}{}
by\_phone returns a User object given a phone number
\begin{quote}\begin{description}
\item[{Parameters}] \leavevmode\begin{itemize}
\item {} 
\textbf{cls} (\emph{class}) -- The calling class

\item {} 
\textbf{phone} (\emph{string}) -- The phone number of the user being searched for

\end{itemize}

\end{description}\end{quote}

\end{fulllineitems}


\end{fulllineitems}



\chapter{Indices and tables}
\label{index:indices-and-tables}\begin{itemize}
\item {} 
\emph{genindex}

\item {} 
\emph{modindex}

\item {} 
\emph{search}

\end{itemize}


\renewcommand{\indexname}{Python Module Index}
\begin{theindex}
\def\bigletter#1{{\Large\sffamily#1}\nopagebreak\vspace{1mm}}
\bigletter{c}
\item {\texttt{cagenix}}, \pageref{dev-overview:module-cagenix}
\item {\texttt{cagenix.patients.models}}, \pageref{dev-patients:module-cagenix.patients.models}
\item {\texttt{cagenix.users.models}}, \pageref{dev-users:module-cagenix.users.models}
\end{theindex}

\renewcommand{\indexname}{Index}
\printindex
\end{document}
